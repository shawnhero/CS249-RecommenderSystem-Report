%%%%%%%%%%%%%%%%%%%%%%%%%%%%%%%%%%%%%%%%%%%%%%%%%%%%%%%%%%%%%%%%%%%%%%%%%%%
%
% Template for a LaTex article in English.
%
%%%%%%%%%%%%%%%%%%%%%%%%%%%%%%%%%%%%%%%%%%%%%%%%%%%%%%%%%%%%%%%%%%%%%%%%%%%

\documentclass{article}

% AMS packages:
\usepackage{amsmath, amsthm, amsfonts}

\usepackage{fullpage}
\usepackage{multirow}
\usepackage{graphicx}


% Theorems
%-----------------------------------------------------------------
\newtheorem{thm}{Theorem}[section]
\newtheorem{cor}[thm]{Corollary}
\newtheorem{lem}[thm]{Lemma}
\newtheorem{prop}[thm]{Proposition}
\theoremstyle{definition}
\newtheorem{defn}[thm]{Definition}
\theoremstyle{remark}
\newtheorem{rem}[thm]{Remark}

% Shortcuts.
% One can define new commands to shorten frequently used
% constructions. As an example, this defines the R and Z used
% for the real and integer numbers.
%-----------------------------------------------------------------
\def\RR{\mathbb{R}}
\def\ZZ{\mathbb{Z}}

% Similarly, one can define commands that take arguments. In this
% example we define a command for the absolute value.
% -----------------------------------------------------------------
\newcommand{\abs}[1]{\left\vert#1\right\vert}

% Operators
% New operators must defined as such to have them typeset
% correctly. As an example we define the Jacobian:
% -----------------------------------------------------------------
\DeclareMathOperator{\Jac}{Jac}

%-----------------------------------------------------------------
\title{Recommender System}
\author{Zijun Xue, Wen Shi and Xu Wu\\
%  \small Dept. Templates and Editors\\
 % \small E12345\\
  %\small Spain
}

\begin{document}
\maketitle

\abstract{In this survey, we reviewed the paper \emph{Simple and Deterministic Matrix Sketching}, which is published in SIGKDD 2013. This paper introduced a cut-edge matrix sketching approach. We first introduce the background and related method of this work. We then describe the main property and algorithm of this method. Finally, we analyze this work and do comparison with other related works based on the accuracy and time complexity.}

%\section{Sample Section, with samples below}
%
%Here goes the text.
%\begin{equation}\label{eq:area}
%  S = \pi r^2
%\end{equation}
%One can refer to equations like this: see equation (\ref{eq:area}). One can also
%refer to sections in the same way: see section \ref{sec:nothing}. Or
%to the bibliography like this: \cite{Cd94}.
%
%\subsection{Subsection}\label{sec:nothing}
%
%More text.
%
%\subsubsection{Subsubsection}\label{sec:nothing2}
%
%More text.

%-----------------------------------------------------------------
% Part One
% Review of Background

\section{Review of Background}


\section{Algorithm}

\begin{thebibliography}
test
\end{thebibliography}


\end{document}
